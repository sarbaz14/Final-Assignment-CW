%In the name of Allah(Gad)
\documentclass[titlepage]{article}
%\usepackage{graphicx} % Required for inserting images
\usepackage{fancyhdr}


\title{
    \textbf{\Huge In the name of Allah}\\
    \vspace{2in}
    \textbf{Computer Workshop}\\
    Final Assignment\\
    Integration of Tools and Practices\\
    \vspace{0.1in}
    \large Iran University of Science and Technology\\
    \large Department of Computer Engin
    \vspace{0.5in}
}


\author{
    \vspace{0.1in}
    Seyyed Mahdi Mousavi\\
    Computer Workshop 02-03\\
    \vspace{0.2in}
}
\date{5 bahman, 1402}



\begin{document}
\pagestyle{fancy}

\maketitle
\tableofcontents
\newpage

\maketitle

\section{Repository Initialization and Commits}
\begin{enumerate}
    \item Opening the GitHub website
    \item Click the "New" button on the top left.
    \item Enter the name of your desired repository in the "Repository name" section.
    \item Insert a brief explanation about the repository in the "Description" section.
    \item Click "Create repository".
    \item Now, in the system, within the desired folder, enter the command "git init".
    \item Then enter the command "git remot add <url>" (replace "url" with the url from the "Code" page of the repository).
\end{enumerate}

\section{GitHub Actions for LaTeX Compilation}
\begin{enumerate}
    \item Creating main.yml file
    \item Placing main.yml file inside the .github/workflows folder
    \item Adding the ".github" folder
    \item Committing changes
    \item Pushing
    \item With main.yml configured correctly, actions should automatically execute once the latest Commit is tagged with a v*.*.* number. Finally, the code is pushed to GitHub with the command "git push origin v*.*.*".
\end{enumerate}

\end{document}

